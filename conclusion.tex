\section{Conclusion}
\label{ch:conclusion}

This paper proposed a model adaptation framework of different models of EDAs for the permutation optimization problems, three distance measure for the RTR and the method of distance-measure adaptation. We stated by means of examples of the permutation problems that, although all solutions are encoded as permutations, their semantics change from one problem to another. Several permutation models using different strategies are illustrated. They showed promising results, but each model can only deal with some parts of the permutation problems effectively. To achieve best performance, choosing dedicated models for specific problem is critical. The semantics of the permutation problems may out of the capability of a model. As a consequence, the model adaptation is essential. For the same reason, the distance-measure adaptation is essential, too.

Experimental results indicates that the method of model adaptation and the method of distance-measure adaptation are promising. Several approaches to achieve model adaptation, three distance measure and the method of distance-measure adaptation were proposed. In order to evaluate the efficiency, they were tested on several well-known permutation problems,including the linear ordering problem, the permutation flow-shop scheduling problem, the capacitated vehicle routing problem and the traveling salesman problem. Experimental results revealed that the multi-armed bandit based model adaptation (MABMA) can achieve better performance than others do. The MABMA can achieve better performance than single model EDAs on complicated permutation problem, and close to the single model EDAs on the problems having dedicated model. Besides, the RTR is effective to improve the performance on the permutation problems.

As for future work, we investigate the techniques learning the needs of different models and distance measures in a permutation segment. Although the model adaptation can deal with more permutation problem, some problems shows more complicated semantics like the vehicle routing problem with time window (VRPTW). In different segment of the permutation in VRPTW, different model is necessary. Also, we are interested in developing the mechanism which can select the best fit model.