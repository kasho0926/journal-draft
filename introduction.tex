\section{Introduction}
\label{ch:introduction}

%介紹 - 1. EDA的發展, 2. Permutation Problem的分歧性, 3. EDA for permutaitons
% 4. MAB 5. Road Map 
Estimation of distribution algorithms (EDAs)~\citep{lozano2006towards, larranaga2002estimation, pelikan2002survey, Hauschild2011} constitute a powerful Evolutionary Algorithm (EA) tool for optimization. The main characteristic of EAs is the use of techniques inspired by the natural evolution of the species. In nature, species change across time; individuals evolve, adapting to the characteristics of the environment. This evolution leads individuals with better characteristics. EDAs are stochastic optimization techniques that explore the space of potential solutions by building and sampling explicit probabilistic models of promising candidate solutions. This model-based approach to optimization has allowed EDAs to solve many large and complex problems. The same idea is translated to the realm of computation, where an individual presents a particular solution for the problem to be solved, a population comprises several individuals, and different operators such as crossover, mutation and selection techniques which are used to make the individuals (solutions) evolve. The most popular reference of these types of algorithms are the Genetic Algorithms (GAs)~\citep{Goldberg:1989:GAS:534133}.

Along the development of GAs, EDAs were introduced in the field of EAs in~\citep{muhlenbein1996recombination}. Different from GAs, EDAs learns a joint probability distribution associated with the set of most promising individuals at each generation, trying to explicitly express the interrelations between the different variables of the problem. Sampling the probabilistic model generated in the previous generation, a new population of solutions for the problem is generated. The algorithm stops iterating and returns the best solution found across the generations when a certain stopping criterion is met, such as maximum number of evaluation (NFE), homogeneous population, or lack of improvement in the solutions.

The general EDA framework begins with the initialization stage, which randomly generates a initial population. Next, the evaluation stage calculates the fitness of each individual. The fitness represents the quality of a individual. The selection stage selects a set of individuals outperforming the others. Then, in the model building stage, the probabilistic model is built with the selected individuals. The sampling stage generates the offspring with the model just built. In the replacement stage, the offspring replace the original population. Then, evaluate the individuals again and repeat this flow until meet the terminate criterion. Algorithm \ref{alg:eda} introduces a pseudo-code of EDAs.

\begin{algorithm}[t]
    generate initial population $P$\;
    \While{not meet the terminate criterion}{
        select population of promising solution $S$\;
        build probabilistic model $M$ for $S$\;
        sample $M$ to generate offspring set $O$\;
        incorporate $O$ into $P$\;
    }
    \caption{Estimation of distribution algorithm (EDA)}
    \label{alg:eda}
\end{algorithm}

Based on this general framework, several EDA approaches have been developed in the last years~\citep{Harik1999, pelikan2005hierarchical, Yu2009}, where each approach learns a specific probabilistic model that conditions the behavior of the EDA from the point if view of complexity and performance. Many works in the literature confirm the good performance of EDAs in the solution of problems from diverse fields. Protein Folding~\citep{santana2008protein}, Capacitated Vehicle Routing Problems~\citep{tsutsui2004solving}, Calibration of Chemical Applications~\citep{mendiburu2006parallel}, Treatment Optimization for Cancer~\citep{brownlee2008application} or Nuclear Reactor Fuel Management Parameter Optimization~\citep{jiang2006estimation} are some examples of many real-world problems where EDA-based approaches were applied to find optimal solutions.

In the past decades, the literature have been interested in the solution of a specific subset of NP-hard optimization problems~\citep{ceberio2012review, bosman2001crossing, tsutsui2002probabilistic, tsutsui2006node, ceberio2011introducing, ceberio2013plackett, pelikan2007dependency}. Particularly, those problems whose solutions can be naturally represented as a permutation. In combinatorics, a permutation is understood as a vector $\sigma = (\sigma_1,..., \sigma_n)$ of the indices ${1,...,n}$ such that $\sigma_i \neq \sigma_j$ for all $i \neq j$. Index $j$ is in position $i$ in $\sigma$ when $\sigma_i = j$.

In this paper, this kind of problems are referred as permutation problem. In some problems, solution quality depends mainly on the relative ordering of pairs of permutation elements and the task is to find a permutation to satisfy a set of relative-ordering constraints; an example problem of this type is job-scheduling~\citep{taillard1993benchmarks, johnson1954optimal}. Sometimes solution quality depends mainly on the pairs of elements that are located next to each other (neighbors), which is a typical feature of the traveling salesman problem~\citep{goldberg1985alleles}and similar problems. Since the semantics differs from problem to problem, many models are proposed for different kind of permutation problems~\citep{tsutsui2002probabilistic, tsutsui2006node, ceberio2011introducing, ceberio2013plackett, tsutsui2006edge, tsutsui2006comparative}. 

In~\citep{tsutsui2002probabilistic}, Tsutsui proposed a model named Edge Histogram Matrix which learns the adjacency of the indices. The experiment shows promising results that algorithms with Edge Histogram Matrix outperform on the traveling salesman problem. In~\citep{tsutsui2006node}, Node Histogram Matrix models the absoluter position of indices which performs well on the quadratic assignment problem. Since a model can only deal with a subset of permutation problems, models keeps being proposed and being design. But the semantics of more permutation problems are complicated, like the permutation flow-shop problem and the capacitated vehicle routing problem with constraints.
 
This paper aims to incorporate different permutation models and make them work together. The proposed framework in this paper makes models cooperate with ease. The technique for multi-armed bandit problems is introduced in the proposed framework. Multi-armed bandit problems have been introduced by Robbins~\citep{robbins1985some} and have since been used extensively to model the tradeoffs faced by an automated agent which aims to gain new knowledge by exploring its environment and to exploit its current, reliable knowledge~\citep{robbins1985some, Kuleshov2000, Mohri2005}.

In order to study the efficiency of the multi-armed bandit based model adaptation, we deal with several different problems, the linear ordering problem, the permutation flow-shop scheduling problem, the capacitated vehicle routing problem, the traveling and so on. we compare the results with those obtained by other EDAs for permutation problem.

The remainder of this paper is organized as follows. In the first section, the permutation problems are defined more specifically. Section \ref{ch:related_works} describes several permutation models in detail. Section \ref{ch:essence_of_adaptation} indicates the essence of model adaptation. Next, Section \ref{ch:methodology} discusses several model adaptation approaches, and lists experiment results. the multi-armed bandit based model adaptation is also introduced in this section. Finally, some conclusions and future work ideas are presented in Section \ref{ch:conclusion}.
